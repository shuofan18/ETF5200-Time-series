\documentclass[10pt,]{article}
\usepackage{lmodern}
\usepackage{amssymb,amsmath}
\usepackage{ifxetex,ifluatex}
\usepackage{fixltx2e} % provides \textsubscript
\ifnum 0\ifxetex 1\fi\ifluatex 1\fi=0 % if pdftex
  \usepackage[T1]{fontenc}
  \usepackage[utf8]{inputenc}
\else % if luatex or xelatex
  \ifxetex
    \usepackage{mathspec}
  \else
    \usepackage{fontspec}
  \fi
  \defaultfontfeatures{Ligatures=TeX,Scale=MatchLowercase}
\fi
% use upquote if available, for straight quotes in verbatim environments
\IfFileExists{upquote.sty}{\usepackage{upquote}}{}
% use microtype if available
\IfFileExists{microtype.sty}{%
\usepackage{microtype}
\UseMicrotypeSet[protrusion]{basicmath} % disable protrusion for tt fonts
}{}
\usepackage[margin=1in]{geometry}
\usepackage{hyperref}
\hypersetup{unicode=true,
            pdfborder={0 0 0},
            breaklinks=true}
\urlstyle{same}  % don't use monospace font for urls
\usepackage{longtable,booktabs}
\usepackage{graphicx,grffile}
\makeatletter
\def\maxwidth{\ifdim\Gin@nat@width>\linewidth\linewidth\else\Gin@nat@width\fi}
\def\maxheight{\ifdim\Gin@nat@height>\textheight\textheight\else\Gin@nat@height\fi}
\makeatother
% Scale images if necessary, so that they will not overflow the page
% margins by default, and it is still possible to overwrite the defaults
% using explicit options in \includegraphics[width, height, ...]{}
\setkeys{Gin}{width=\maxwidth,height=\maxheight,keepaspectratio}
\IfFileExists{parskip.sty}{%
\usepackage{parskip}
}{% else
\setlength{\parindent}{0pt}
\setlength{\parskip}{6pt plus 2pt minus 1pt}
}
\setlength{\emergencystretch}{3em}  % prevent overfull lines
\providecommand{\tightlist}{%
  \setlength{\itemsep}{0pt}\setlength{\parskip}{0pt}}
\setcounter{secnumdepth}{0}
% Redefines (sub)paragraphs to behave more like sections
\ifx\paragraph\undefined\else
\let\oldparagraph\paragraph
\renewcommand{\paragraph}[1]{\oldparagraph{#1}\mbox{}}
\fi
\ifx\subparagraph\undefined\else
\let\oldsubparagraph\subparagraph
\renewcommand{\subparagraph}[1]{\oldsubparagraph{#1}\mbox{}}
\fi

%%% Use protect on footnotes to avoid problems with footnotes in titles
\let\rmarkdownfootnote\footnote%
\def\footnote{\protect\rmarkdownfootnote}

%%% Change title format to be more compact
\usepackage{titling}

% Create subtitle command for use in maketitle
\newcommand{\subtitle}[1]{
  \posttitle{
    \begin{center}\large#1\end{center}
    }
}

\setlength{\droptitle}{-2em}
  \title{}
  \pretitle{\vspace{\droptitle}}
  \posttitle{}
  \author{}
  \preauthor{}\postauthor{}
  \date{}
  \predate{}\postdate{}


\begin{document}

\begin{aligned}#ETF5200 Applied time series econometrics Project 1\end{aligned}

\subsection{Part I}\label{part-i}

\subsubsection{Question 1}\label{question-1}

\(c_t\) is the logarithm of the per-capital consumption expenditure,
\(i_t\) is the logarithm of the per-capital disposable income, \(p_t\)
is the logarithm of GDP, \(r_t\) is the real interest rate.

All numbers are rounded to four decimal places.

\begin{longtable}[]{@{}lllll@{}}
\caption{OLS estimates for each series, p.value is shown in the bracket
below each estimates, most of them are insignificant based on 5\%
significance level}\tabularnewline
\toprule
series & variance & alpha & beta & gamma\tabularnewline
\midrule
\endfirsthead
\toprule
series & variance & alpha & beta & gamma\tabularnewline
\midrule
\endhead
ct & 0.0001 & 0.0743 & 0.0001 & -0.0091\tabularnewline
p.value & & (0.4646) & (0.5941) & (0.5266)\tabularnewline
it & 0.0001 & 0.0873 & 0.0001 & -0.0107\tabularnewline
p.value & & (0.3252) & (0.487) & (0.3875)\tabularnewline
pt & 0.0001 & 0.108 & 0.0001 & -0.013\tabularnewline
p.value & & (0.3119) & (0.45) & (0.3602)\tabularnewline
rt & 0.8986 & 0.2328 & -0.0007 & -0.0835\tabularnewline
p.value & & (0.1251) & (0.5662) & (0.0059)\tabularnewline
\bottomrule
\end{longtable}

\subsubsection{Question 2}\label{question-2}

To do unit root tests, we first need to determine whether there is trend
or not. And for ADF test, we also need to find the proper lags.

Therefore, we plot each series to check the trend component. As shown in
the figure 1 (Appendix), there are clear trend in ct, it and pt; but no
trend in rt. So the hypothesis tests will be ``has both unit root and
trend'' vs. ``trend stationary'' for ct, it and pt; ``has unit root''
vs. ``stationary'' in case of rt.

As for proper lags, we let ``adf.test'' function in R to automatically
choose one, and it chooses 6, 6, 6, 5 for ct, it, pt, and rt
respectively. Then we use Durbin-Watson test to test serial correlation
in the associated four sets of residuals. All four p-values are not
showing enough evidence to reject the null. So we think the lags chosen
by ``adf.test'' are good enough based on 5\% significance level. We will
use them for ADF test directly.

\begin{longtable}[]{@{}lllll@{}}
\caption{lags chosen by adf.test and p.value from Durbin-Watson
test}\tabularnewline
\toprule
& ct & it & pt & rt\tabularnewline
\midrule
\endfirsthead
\toprule
& ct & it & pt & rt\tabularnewline
\midrule
\endhead
lags & 6.00 & 6.00 & 6.0 & 5.00\tabularnewline
p.value & 0.41 & 0.44 & 0.4 & 0.29\tabularnewline
\bottomrule
\end{longtable}

\begin{longtable}[]{@{}lllllll@{}}
\caption{Three unit root tests, ADF and PP always have same conclusions,
KPSS makes two different decisions}\tabularnewline
\toprule
& ADF & ADF & PP & PP & KPSS & KPSS\tabularnewline
\midrule
\endfirsthead
\toprule
& ADF & ADF & PP & PP & KPSS & KPSS\tabularnewline
\midrule
\endhead
& conclusion & p.value & conclusion & p.value & conclusion &
p.value\tabularnewline
ct & trend and unit root & 0.1 & trend and unit root & 0.21 & trend and
unit root & 0.01\tabularnewline
it & trend stationary & 0.02 & trend stationary & 0.03 & trend and unit
root & 0.01\tabularnewline
pt & trend and unit root & 0.32 & trend and unit root & 0.32 & trend and
unit root & 0.01\tabularnewline
rt & unit root & 0.09 & unit root & 0.11 & stationary &
0.08\tabularnewline
\bottomrule
\end{longtable}

From the table, we can see that ADF and PP always make same decisions
for these four series. Since there are no serial correlation left in the
residuals for ADF test (given DW test), and the most important feature
of PP test is to correct the calculation of standard deviation for the
test statistic when there are serial correlation in the residuals, we
expect ADF and PP to performs similarly in this case, so this result
meets our expectation. What's more, KPSS differs from those two tests
for ``it'' and ``rt''. Because KPSS test also expect the residuals to be
i.i.d and it uses different lags with ADF test, so the residuals in KPSS
test may not be i.i.d. With this possible violation of assumption, we
think KPSS's conclusions are unreliable in this case.

\subsection{Part II}\label{part-ii}

\subsubsection{The main idea proposed}\label{the-main-idea-proposed}

\subsubsection{The main techniques used}\label{the-main-techniques-used}

\subsubsection{The main data used}\label{the-main-data-used}

\subsubsection{The main results
obtained}\label{the-main-results-obtained}

\subsubsection{The conclusions made in the
paper}\label{the-conclusions-made-in-the-paper}

\subsection{Appendix}\label{appendix}

\begin{figure}
\centering
\includegraphics{new_asg2_files/figure-latex/ctplot-1.pdf}
\caption{Time plot of four series}
\end{figure}


\end{document}
